 \documentclass[11 pt]{article}
\usepackage{amssymb,amsmath, amsthm, graphics, subfig, listings, graphicx}
%\usepackage{natbib}
%formatting that is good for editing (it double spaces)
\renewcommand{\baselinestretch}{1.4}
\textwidth 6in \textheight 9in \hoffset -0.30in \topmargin -0.45in
\interfootnotelinepenalty=10000 %keeps footnotes on a single page

%comment out
%----------------------
\newcommand{\commentout}[1]{}
%--------------------------

%%%%%%%%%%%%%%%%%%%%%%%%%%%%%%%%%%%%%%%%%%%%%%%%%%%%%%%%%%%%%%%%%%%%%%%%%%%%%%%%%
%PART I. GENERAL COMMANDS

%%%%%%%%%%%%%%%%%%%%%%%%%%%%%%%%%%%%%%%%%%%%%%%%%%%%%%%%%%%%%%%%%
% I. NUMBERING AND ENVIRONMENTS FOR THEOREMS, PROPOSITIONS, LEMMAS, AND EQUATIONS

%A. Actual Theorems
\newtheorem{ass}{Assumption}
\newtheorem{prop}{Proposition}
\newtheorem{fact}{Fact}
\newtheorem{lem}{Lemma}
\newtheorem{claim}{Claim}
\newtheorem{thm}{Theorem}
\newtheorem{cor}{Corollary}
\newtheorem{con}{Conjecture}
\newtheorem{defn}{Definition} %use \textbf{} to set it off
\newtheorem{rem}{Remark}
\newtheorem{rec}{Recall}
\newtheorem{problem}{Problem}
\newtheorem{example}{Example}
\newtheorem{note}{Note}
\newtheorem{question}{Question}
\newtheorem{ques}{\textcolor{red}{Question}}
\newtheorem{com}{\textcolor{red}{Comment}}
\newtheorem{todo}{\textcolor{red}{To Do}}

%B. Special Referencing

%B1. Built in ones

%1. \eqref (equations, in standard package)
 %%2. in commath:
%a)\thmref
%b)\exref (example)
%c) \defnref
%d) \lemref
%e)\propref
%f)\remref
%g)\assref
%h)\colref - corollary
%also figures, section, appendix

%B2. I try to define
%1. fact KEY need dollar signs for it work $\factref{fact:varW}$
\newcommand{\factref}[1]{ \text{Fact~\ref{#1}} }

%%%%%%%%%%%%%%%%%%%%%%%%%%%%%%%%%%%%%%%%%%%%%%%%%%%%%%%%%%%%%%%%%

%Bold Greek
%%%%%%%%%%%%%%%%%%%%%%%%%%%%%%%%%%%%%%%%%%%%%%%%%%%%%%%%%%%%%%%%%
\def\bbeta{\mbox{\boldmath $\beta$}}
\def\bmu{\mbox{\boldmath $\mu$}}
\def\etab{\mbox{\boldmath $\eta$}}
\def\balpha{\mbox{\boldmath $\alpha$}}
\def\btau{\mbox{\boldmath $\tau$}}
\def\bDelta{\mbox{\boldmath $\Delta$}}
\def\bGamma{\mbox{\boldmath $\Gamma$}}
\def\bgamma{\mbox{\boldmath $\gamma$}}
\def\bOmega{\mbox{\boldmath $\Omega$}}
\def\bPsi{\mbox{\boldmath $U$}}
\def\bpsi{\mbox{\boldmath $\mu$}}
\def\bXi{\mbox{\boldmath $\Xi$}}
\def\bxi{\mbox{\boldmath $\xi$}}
\def\bSigma{\mbox{\boldmath $\Sigma$}}
\def\bLambda{\mbox{\boldmath $\Lambda$}}
\def\btheta{\mbox{\boldmath $\theta$}}
\def\bDelta{\mbox{\boldmath $\Delta$}}
\def\bTheta{\mbox{\boldmath $\Theta$}}
\def\etaz{\mbox{\boldmath $\eta$}}

\def\boldX{\mbox{\boldmath $X$}}
\def\boldx{\mbox{\boldmath $X$}}
%%%%%%%%%%%%%%%%%%%%%%%%%%%%%%%%%%%%%%%%%%%%%%%%%%%%%%%%%%%%%%%%%

%%%%%%%%%%%%%%%%%%%%%%%%%%%%%%%%%%%%%%%%%%%%%%%%%%%%%%%%%%%%%%%%%
% III. SHORTCUTS

%A. Greek symbols
\newcommand{\be}{\begin{eqnarray*}}
\newcommand{\ee}{\end{eqnarray*}}
\newcommand{\ff}{\infty}
\newcommand{\ra}{\rightarrow}
\newcommand{\ep}{\epsilon}
\newcommand{\ga}{\gamma}
\newcommand{\al}{\alpha}
\newcommand{\la}{\lambda}
\newcommand{\si}{\sigma}
\renewcommand{\th}{\theta}
\newcommand{\Epos}{E_{\theta|\boldX}}
\newcommand{\Ej}{E_{\theta,\boldX}}
%B. Other Math Symbols

%1. transpose, you want a consistent use, bc you could use t,T or \prime
\def\tran{\mathop{ t }}

%C. Commands
\newcommand{\xra}[1]{\mathop{ \xrightarrow{#1} }}

%D. Fences puts items in the correct fence sizes (parantheses, brackets, etc..)
% (get rid of? look at commath package?)
%fp = fence parentheses
\newcommand{\fp}[1]{ \mathop{ \left( #1 \right) } }
%fb = fence brackets
\newcommand{\fb}[1]{ \mathop{ \left[ #1 \right] } }
%fbr = fence braces meaning \{
\newcommand{\fbr}[1]{ \mathop{ \left\{ #1 \right\} } }

%%%%%%%%%%%%%%%%%%%%%%%%%%%%%%%%%%%%%%%%%%%%%%%%%%%%%%%%%%%%%%%%%


%%%%%%%%%%%%%%%%%%%%%%%%%%%%%%%%%%%%%%%%%%%%%%%%%%%%%%%%%%%%%%%%%
%V. Other Commands

 %command that is called by \Title{Text} and this centers it.
\newcommand{\Title}[1]{\begin{center}{\Large \bf #1} \end{center}}
\newcommand{\hype}[1]{\mathcal{H}_{#1}}
\newcommand{\pval}{\text{pvalue}}
\newcommand{\sediff}{\text{SE}_{\bar{Y}_1 - \bar{Y}_2}}
%%%%%%%%%%%%%%%%%%%%%%%%%%%%%%%%%%%%%%%%%%%%%%%%%%%%%%%%%%%%%%%%%
%END PART I

\newcommand{\fth}{f_{\th}}
\newcommand{\fths}{f_{\th,s}}

\begin{document}
\Title{EM algorithm for brood size sampling.}

\Title{John Davis, Bret Hanlon}

\section{Setup}
Let $X$ be the sampled brood size vector for a draw of size $n$ from one generation of a branching process with a shifted power series offspring distribution denoted by $\fth$. The size-biased data $X$ follow a related distribution $\fths$ when no single family is duplicated in the sample. 

The data are thus approximately 
\[
\Pi_{i=1}^n \fths(x_i).
\]
We can re-represent the data vector $X$ as 
\[
(X_1,...,X_n) = (Y_i,n_i : i=1,...,\bar{n})
\]
where $Y_i$ is the unique values $X$ takes, $n_i$ is the number of such values for a particular $Y_i$, and $\bar{n}$ is the number of unique values of $X$. Suppose, now, that we observe the number of distinct families that compose each pair $(Y_i,n_i)$. Let this number be $Z_i$. Evidently
\[
Z_i \in S_i:= \fbr{ [n_i / y_i],...,n_i},
\]
where $[\cdot]$ denote the ceiling function.  Then, 
\[
P(X|Z,\th) = \Pi_{i=1}^{\bar{n}} \fths^{Z_i}(Y_i)
\]
exactly instead of
\[
\Pi_{i=1}^{\bar{n}} \fths^{n_i}(Y_i)
\]
which is incorrect because of the possibility of resampling the same family. Note that this doesn't account for the problem that the sampling is done with respect to a finite population.  Assume a uniform prior in the support of $S_i$ for $Z_i$. Then
\[
P(Z_i|X,\th) = P(Z_i|Y_i,n_i,\th) \propto P(Y_i|Z_i,n_i,\th) = \fths^{Z_i}(Y_i)
\]
with normalization constant $A_i$, say.

For the remainder of this document, for simplicity, we assume that the offspring distribution follows a shifted geometric distribution, given by
% #likelihood of shifted geometric
% model.likelihood = function(x,theta) {
%   #hardcoded geometric dist shifted by +1
%   return(theta^(x)*(1-theta)/theta)
% }

% model.likelihood.s = function(x,theta) {
%   #hardcoded geometric dist shifted by +1
%   return( x * theta^x * (1 - theta)^2 / theta )
% }
\[
\fth(x) = \th^{x -1} (1 - \th),
\]
and its size-biased version is given by
\[
\fths(x) = x \th^{x - 1} (1 - \th)^2
\]

\section{EM Algorithm}

\subsection{EM Algorithm for the Size-Biased Distribution}

Let
\[
Q(\th | \th_i) = E_{Z|X,\th_i} \log L(\th,X,Z)
\]
where $\th_i$ is the parameter value of the previous iteration. Here,
\[
L(\th,X,Z) = L(X | \th, Z) L(Z | \th) \propto L(X | \th, Z).
\] 
Again, we have the problem of setting the prior on $Z$ to be uniform; in other words, we ignore the distribution $P(Z | \th)$ and $P(Z | X, \th)$ whenever possible.


Then maximize $Q(\th|\th_i)$ and set $\th_{i+1}$ equal to the maximizer. Repeat this until convergence.

The explicit form of $Q$ is given here
\[
Q(\th | \th_i) = \sum_{j=1}^{\bar{n}} \sum_{z \in S_j} z \log \fths (y_i) f^z_{\th_i,s}(y_i) / A_i.
\]
This is the expectation calculation when we assume that the $Z | X, \th$ is uniform on the support of $Z_i$, denoted by $S_i$.


\section{Gibbs Sampler}

The EM algorithm is somehwat artlessly applied because it ignores two problems: (1) the population size is finite, and (2) the number of unique families $Z$ is not uniform on its support. A hierarchical model is proposed which allows for an exact calculation of the distribution of $Z_i$. Suppose brood-sized sampling in a branching process with two generations, parent and child. In the parent generation, assume $N$ objects. Assume a sample of size $n$ from the child generation, and re-write the sample as before using $(Y_i,n_i : i=1,...,\bar{n})$. Then consider the following model:
\begin{align*}
\th &\sim Beta(\alpha, \beta) \\
(Y_i, n_i) = (X_1,...,X_n) | Z_i, \th &\sim \Pi_{i=1}^{\bar{n}} \fths^{Z_i}(Y_i) \\
Z_i | N, Y_i, n_i, \th &\sim H(Z_i | N, Y_i, n_i, \th)
\end{align*}
where $N$ is estimated. We define $H$ now. Consider one of the values $Y_i$, which appeared $n_i$ times in the sample. The number of unique parent families for these $n_i$ values is given by $S_i$. We also know there are at least $Y_i  Z_i$ parent nodes giving off the value $Y_i$. Given this collection of parent nodes (perhaps there are $N_i$ of them), the distribution of $Z_i$ is related the the multivariate hypergemetric distribution. Each parent taking the value $Y_i$ should be regarded as a different color, with number of balls equal to $Y_i$. Then draw $n_i$ balls and count the number of colors. This is the value of $Z_i$, and $H$ is the corresponding conditional probability distribution. 

The Gibbs sampler uses an iteration step on the full conditional probabilities, which are derived here.

\begin{itemize}
  \item First, we need
\begin{align*}
P(\th | Y_i, Z_i) = P(Y_i, Z_i | \th) P(\th) = P(Y_i | Z_i, \th) P(Z_i | \th) P(\th)
\end{align*}
I assume that $Z$ and $\th$ are independent, giving
\begin{equation} \label{firstcond}
P(\th | Y_i, Z_i) = Beta(\alpha + \sum_{i = 1}^{\bar{n}} Z_i(Y_i - 1), \beta + 2 \sum_{i = 1}^{\bar{n}} Z_i)
\end{equation}

 \item Second, we need the parent generation given the rest of the parameters. Denote the number of offspring of each parent by $O_1,...,O_N$. Certainly these $O_i$ will, in part, be made up of $Z_1$ of $Y_1$, $Z_2$ of $Y_2$, and so on. The remaining $N - \sum_i^{\bar{n}}Z_i$ are drawn from $\fth$.

We could alternatively (and perhaps more accurately) write
\begin{align*}
P(\th | Y, Z, O) = P(\th | O) = P(O | \th) P(\th) = \\\Pi_{i = 1}^N \fth(o_i) \times Beta(\alpha,\beta) = Beta(\alpha + \sum O_i - N, \beta + N)
\end{align*}
instead of Equation \eqref{firstcond}.

 \item Third, we need
\[
P(Z_i | Y_i, \th, O_i).
\]
This is, as described before, related to the multivariate hypergeometric distribution. In particular, draw $n_i$ balls from an urn with $\sum_j O_j 1(O_j = Y_i)$ balls in it. The nonzero terms of the sum give the number of balls of each color. Then $Z_i$ is the number of colors in the sample. 
\end{itemize}

Using a Gibbs sampler on the condition probabilities above, we can obtain the posterior for $\th$. 

\end{document}
