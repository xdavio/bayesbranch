 \documentclass[11 pt]{article}
\usepackage{amssymb,amsmath, amsthm, graphics, subfig, listings, graphicx}
%\usepackage{natbib}
%formatting that is good for editing (it double spaces)
\renewcommand{\baselinestretch}{1.4}
\textwidth 6in \textheight 9in \hoffset -0.30in \topmargin -0.45in
\interfootnotelinepenalty=10000 %keeps footnotes on a single page

%comment out
%----------------------
\newcommand{\commentout}[1]{}
%--------------------------

%%%%%%%%%%%%%%%%%%%%%%%%%%%%%%%%%%%%%%%%%%%%%%%%%%%%%%%%%%%%%%%%%%%%%%%%%%%%%%%%%
%PART I. GENERAL COMMANDS

%%%%%%%%%%%%%%%%%%%%%%%%%%%%%%%%%%%%%%%%%%%%%%%%%%%%%%%%%%%%%%%%%
% I. NUMBERING AND ENVIRONMENTS FOR THEOREMS, PROPOSITIONS, LEMMAS, AND EQUATIONS

%A. Actual Theorems
\newtheorem{ass}{Assumption}
\newtheorem{prop}{Proposition}
\newtheorem{fact}{Fact}
\newtheorem{lem}{Lemma}
\newtheorem{claim}{Claim}
\newtheorem{thm}{Theorem}
\newtheorem{cor}{Corollary}
\newtheorem{con}{Conjecture}
\newtheorem{defn}{Definition} %use \textbf{} to set it off
\newtheorem{rem}{Remark}
\newtheorem{rec}{Recall}
\newtheorem{problem}{Problem}
\newtheorem{example}{Example}
\newtheorem{note}{Note}
\newtheorem{question}{Question}
\newtheorem{ques}{\textcolor{red}{Question}}
\newtheorem{com}{\textcolor{red}{Comment}}
\newtheorem{todo}{\textcolor{red}{To Do}}

%B. Special Referencing

%B1. Built in ones

%1. \eqref (equations, in standard package)
 %%2. in commath:
%a)\thmref
%b)\exref (example)
%c) \defnref
%d) \lemref
%e)\propref
%f)\remref
%g)\assref
%h)\colref - corollary
%also figures, section, appendix

%B2. I try to define
%1. fact KEY need dollar signs for it work $\factref{fact:varW}$
\newcommand{\factref}[1]{ \text{Fact~\ref{#1}} }

%%%%%%%%%%%%%%%%%%%%%%%%%%%%%%%%%%%%%%%%%%%%%%%%%%%%%%%%%%%%%%%%%

%Bold Greek
%%%%%%%%%%%%%%%%%%%%%%%%%%%%%%%%%%%%%%%%%%%%%%%%%%%%%%%%%%%%%%%%%
\def\bbeta{\mbox{\boldmath $\beta$}}
\def\bmu{\mbox{\boldmath $\mu$}}
\def\etab{\mbox{\boldmath $\eta$}}
\def\balpha{\mbox{\boldmath $\alpha$}}
\def\btau{\mbox{\boldmath $\tau$}}
\def\bDelta{\mbox{\boldmath $\Delta$}}
\def\bGamma{\mbox{\boldmath $\Gamma$}}
\def\bgamma{\mbox{\boldmath $\gamma$}}
\def\bOmega{\mbox{\boldmath $\Omega$}}
\def\bPsi{\mbox{\boldmath $U$}}
\def\bpsi{\mbox{\boldmath $\mu$}}
\def\bXi{\mbox{\boldmath $\Xi$}}
\def\bxi{\mbox{\boldmath $\xi$}}
\def\bSigma{\mbox{\boldmath $\Sigma$}}
\def\bLambda{\mbox{\boldmath $\Lambda$}}
\def\btheta{\mbox{\boldmath $\theta$}}
\def\bDelta{\mbox{\boldmath $\Delta$}}
\def\bTheta{\mbox{\boldmath $\Theta$}}
\def\etaz{\mbox{\boldmath $\eta$}}

\def\boldX{\mbox{\boldmath $X$}}
\def\boldx{\mbox{\boldmath $X$}}
%%%%%%%%%%%%%%%%%%%%%%%%%%%%%%%%%%%%%%%%%%%%%%%%%%%%%%%%%%%%%%%%%

%%%%%%%%%%%%%%%%%%%%%%%%%%%%%%%%%%%%%%%%%%%%%%%%%%%%%%%%%%%%%%%%%
% III. SHORTCUTS

%A. Greek symbols
\newcommand{\be}{\begin{eqnarray*}}
\newcommand{\ee}{\end{eqnarray*}}
\newcommand{\ff}{\infty}
\newcommand{\ra}{\rightarrow}
\newcommand{\ep}{\epsilon}
\newcommand{\ga}{\gamma}
\newcommand{\al}{\alpha}
\newcommand{\la}{\lambda}
\newcommand{\si}{\sigma}
\renewcommand{\th}{\theta}
\newcommand{\Epos}{E_{\theta|\boldX}}
\newcommand{\Ej}{E_{\theta,\boldX}}
%B. Other Math Symbols

%1. transpose, you want a consistent use, bc you could use t,T or \prime
\def\tran{\mathop{ t }}

%C. Commands
\newcommand{\xra}[1]{\mathop{ \xrightarrow{#1} }}

%D. Fences puts items in the correct fence sizes (parantheses, brackets, etc..)
% (get rid of? look at commath package?)
%fp = fence parentheses
\newcommand{\fp}[1]{ \mathop{ \left( #1 \right) } }
%fb = fence brackets
\newcommand{\fb}[1]{ \mathop{ \left[ #1 \right] } }
%fbr = fence braces meaning \{
\newcommand{\fbr}[1]{ \mathop{ \left\{ #1 \right\} } }

%%%%%%%%%%%%%%%%%%%%%%%%%%%%%%%%%%%%%%%%%%%%%%%%%%%%%%%%%%%%%%%%%


%%%%%%%%%%%%%%%%%%%%%%%%%%%%%%%%%%%%%%%%%%%%%%%%%%%%%%%%%%%%%%%%%
%V. Other Commands

 %command that is called by \Title{Text} and this centers it.
\newcommand{\Title}[1]{\begin{center}{\Large \bf #1} \end{center}}
\newcommand{\hype}[1]{\mathcal{H}_{#1}}
\newcommand{\pval}{\text{pvalue}}
\newcommand{\sediff}{\text{SE}_{\bar{Y}_1 - \bar{Y}_2}}
%%%%%%%%%%%%%%%%%%%%%%%%%%%%%%%%%%%%%%%%%%%%%%%%%%%%%%%%%%%%%%%%%
%END PART I

\newcommand{\fth}{f_{\th}}

\begin{document}
\Title{Technical Notes for Bayesian MHDE Brood Size Sampling}

\Title{John Davis, Bret Hanlon}


Let the $H$-posterior be given by
\[
P(\th | g_n) \propto \exp \fbr{ -n H(\fth,g_n) } \pi(\th)
\]
where
\begin{itemize}
  \item $\pi$ is the prior
  \item $g_n$ is the density estimator given by
\[
g_n(x) = \left ( \sum_{i=1}^n X_i^{-1} \right )^{-1} \sum_{i=1}^n \frac{1(X_i=x)}{x}, \text{ and}
\]
  \item $\fth$ is the power series offspring distribution with support $1,2,...$.
\end{itemize}
Following (hooker), 
\[
H(\fth,g_n) = \int G(\delta) \fth
\]
where $\delta = g_n / \fth - 1$. It is required that $G(0)=0, G'(0)=0, G''(0)=1$. Some algebra demonstrates that in the case of the Hellinger Distance, 
\[
G(\delta) = 2[(\delta + 1)^{1/2} - 1]^2
\]
and
\[
G(\delta)\fth = 4 - 4 \sum \sqrt{\fth g_n}.
\]
Let $\bar{n}$ be the number of unique values in the sample of brood sizes $X_1,...,X_n$. Enumerate these unique values by $X'_1,..., X'_{\bar{n}}$. Let $n_j = \sum_{i=1}^n 1(X'_j = X_i)$. Noting $g_n(x)$ is positive only on $X'_j$, we can then write
\begin{align*}
-n HD(g_n,\fth) &= 4n \fbr{ \sum_x \sqrt{\fth g_n} - 1 } \\
&= 4n \fbr{ \sum_x \sqrt{\fth \left ( \sum_{i=1}^n X_i^{-1} \right )^{-1} \sum_{i=1}^n \frac{1(X_i=x)}{X_i} } - 1 } \\
&= 4n \fbr{ \sum_{j=1}^{\bar{n}} \sqrt{\fth   \frac{ n_j(X'_j)^{-1}}{\sum_{k=1}^{\bar{n}} n_k (X'_k)^{-1}} }   - 1 }. \\
\end{align*}
This admits alternative expressions for the $H$-posterior:
\[
P(\th | g_n ) \propto \exp \fbr{ 4n \sum_{j=1}^{\bar{n}} \sqrt{\fth   \frac{ n_j(X'_j)^{-1}}{\sum_{k=1}^{\bar{n}} n_k (X'_k)^{-1}} } } \pi(\th)
\]
and
\[
P(\th | g_n ) \propto \left \{ \Pi_{j=1}^{\bar{n}}  \exp \fbr{ \fth   \frac{ n_j(X'_j)^{-1}}{\sum_{k=1}^{\bar{n}} n_k (X'_k)^{-1} } } \right \}^{2n} \pi(\th) 
\]

\section*{Metropolis-Hastings Algorithm}

Define the jump function
\[
J(\th|\th') = logit^{-1}(N(\th',C^2))
\]
where $C$ is a tuning parameter and $N$ is the pdf of a normal distribution with mean $\th'$ and variance $C^2$. Then, accept proposed jump $\th \sim J(\cdot | \th')$ with probability 
\[
\min \fbr{ \frac{P(\th|g_n) J(\th'|\th)}{P(\th'|g_n) J(\th|\th')}  , 1 }
\]


\end{document}
